% Umbenennen der Referenz
\renewcommand{\lstlistingname}{Code-Snipped}% Listing -> Algorithm
\renewcommand{\lstlistlistingname}{Verzeichniss der \lstlistingname e}%
\crefname{listing}{\lstlistingname}{\lstlistingname e}  
\Crefname{listing}{\lstlistingname}{\lstlistingname e}


%% In diesem File werden alle Konfigurationen wie bspw. eine Programierumgebung konfiguriert
\definecolor{codegreen}{rgb}{0,0.6,0}
\definecolor{codegray}{rgb}{0.5,0.5,0.5}
\definecolor{codepurple}{rgb}{0.58,0,0.82}
\definecolor{backcolour}{rgb}{0.95,0.95,0.95}

\definecolor{lstRed}{rgb}{0.88,0.11,0.19} 
\definecolor{lstBlue}{rgb}{0,0,0.39}
\definecolor{lstGreen}{rgb}{0,0.55,0.27}
\definecolor{lstCyan}{rgb}{0.0,0.6,0.6}
\definecolor{lstWhite}{rgb}{0.99,0.99,0.99}
\definecolor{lstBlack}{rgb}{0.0,0.0,0.0}
\definecolor{lstGrey}{rgb}{0.39,.39,0.39}

\definecolor{CommentGreen}{rgb}{0.1,0.35,0.1}
\definecolor{Background}{rgb}{0.97,0.97,0.97}
\definecolor{lrow}{rgb}{0.914,0.918,0.922}
\definecolor{drow}{rgb}{0.725,0.745,0.769}

\lstloadlanguages{C,C++,csh,Java,SQL,Python,bash}

%%

\lstset{
	language=csh,
	basicstyle=\footnotesize\sffamily,
	numbers=left,
	numberstyle=\tiny,
	numbersep=5pt,
	tabsize=2,
	extendedchars=true,
	breaklines=true,
	frame=b,
	stringstyle=\color{lstBlack}\sffamily,
	showspaces=false,
	showtabs=false,
	xleftmargin=17pt,
	framexleftmargin=17pt,
	framexrightmargin=5pt,
	framexbottommargin=4pt,
	commentstyle=\color{lstGrey},
	morecomment=[l]{//}, %use comment-line-style!
	morecomment=[l]{///}, %use comment-line-style!
	morecomment=[s]{/*}{*/}, %for multiline comments
	showstringspaces=false,
	morekeywords={ abstract, event, new, struct,
		as, explicit, null, switch,
		base, extern, object, this,
		bool, false, operator, throw,
		break, finally, out, true,
		byte, fixed, override, try,
		case, float, params, typeof,
		catch, for, private, uint,
		char, foreach, protected, ulong,
		checked, goto, public, unchecked,
		class, if, readonly, unsafe,
		const, implicit, ref, ushort,
		continue, in, return, using,
		decimal, int, sbyte, virtual,
		default, interface, sealed, volatile,
		delegate, internal, short, void,
		do, is, sizeof, while,
		double, lock, stackalloc,
		else, long, static,
		enum, namespace, string},
	keywordstyle=\color{lstBlue},
	identifierstyle=\color{lstBlack},
	backgroundcolor=\color{lstWhite},
}

\lstset{
	language=C++,
	upquote=true, frame=single,
	basicstyle=\small\ttfamily,
	backgroundcolor=\color{Background},
	keywordstyle=[1]\color{blue}\bfseries,
	keywordstyle=[2]\color{purple},
	keywordstyle=[3]\color{black}\bfseries,
	identifierstyle=,
	commentstyle=\usefont{T1}{pcr}{m}{sl}\color{CommentGreen}\small,
	stringstyle=\color{purple},
	showstringspaces=false, tabsize=5,
	morekeywords={properties,methods,classdef},
	morekeywords=[2]{handle},
	morecomment=[l][\color{blue}]{...},
	numbers=none, firstnumber=1,
	numberstyle=\tiny\color{blue},
	stepnumber=1, xleftmargin=10pt, xrightmargin=10pt
}

\lstset{
	language= Python,
	backgroundcolor=\color{backcolour},   
	commentstyle=\color{codegreen},
	keywordstyle=\color{magenta},
	numberstyle=\tiny\color{codegray},
	stringstyle=\color{codepurple},
	basicstyle=\ttfamily\footnotesize,
	breakatwhitespace=false,         
	breaklines=true,                 
	captionpos=b,                    
	keepspaces=true,                 
	numbers=left,                    
	numbersep=5pt,                  
	showspaces=false,                
	showstringspaces=false,
	showtabs=false,                  
	tabsize=4
}


\lstset{literate=% Allow for German characters in lstlistings.
	{Ö}{{\"O}}1
	{Ä}{{\"A}}1
	{Ü}{{\"U}}1
	{ß}{{\ss}}2
	{ü}{{\"u}}1
	{ä}{{\"a}}1
	{ö}{{\"o}}1
}


%tikz
\usetikzlibrary{arrows.meta,positioning}