%% Options
% Hier kommen alle Pakete hin

%% Deutesche Anpassungen
%
\usepackage[ngerman]{babel} 
\usepackage[utf8]{inputenc}
\usepackage[T1]{fontenc}    

%% Schrift Anpassungen
% Latin Modern Schriftart
\usepackage{lmodern}	%Latin Modern Schriftart
\renewcommand{\sfdefault}{\rmdefault}
\setkomafont{sectioning}{\rmfamily \bfseries} %Schriftart ändern
\setkomafont{descriptionlabel}{\rmfamily \bfseries}

%% Design Anpassungen
% Header Style für KOMA Script
\usepackage{scrpage2}
% Design Anpassungen
\usepackage{geometry}
% Optischer Randausgleich
\usepackage{microtype}	

%% Floats Optimierungen
% Behaltet floats im definierten bereich
\usepackage[section]{placeins}
\usepackage{float}


%% Ausrichtung
%
%\usepackage{minipage}
\usepackage{multicol}
% Für einzelne Seiten Querformat
\usepackage{lscape} 
\usepackage{placeins}

%% Bilder
%Zum Laden von Grafiken
\usepackage{graphicx}
\usepackage[justification=centerlast]{caption}
%Teilabbildungen in einer Abbildung
%\usepackage{subfigure}
%Teilabbildungen in einer Abbildung
%Besser als subfigure oder subfig
\usepackage{subcaption}



%% Listen
%
\usepackage{mdwlist}
\usepackage{paralist}
\usepackage{enumitem}
\renewcommand\labelitemi{--}

%% Tabellen
%
\usepackage{array}
\usepackage{tabularx}
\usepackage{booktabs}
% Für Tabellen, die eine Seite überschreiten
\usepackage{longtable} 
\usepackage{multirow}
\usepackage{makecell, boldline}
% Mit diesem Kommando kann in einer Tabelle die Zeile umgebrochen werden
% ... & \specialcell{aaa\\bbb} & \specialcell{aaa\\bbb} \\
\newcommand{\specialcell}[2][c]{%
	\begin{tabular}[#1]{@{}l@{}}#2\end{tabular}}

%% Mathe
%
\usepackage{amsmath} 
\usepackage{amsfonts}
\usepackage{amsthm}
%SI Einheiten usw.
%\usepackage[binary-units]{siunitx}

%% Code
\usepackage{listings}
%%
%\lstdefinestyle{tt}{basicstyle=\small\ttfamily,keywordstyle=\bfseries,language=[LaTeX]{TeX}}
%\lstdefinestyle{rm}{basicstyle=\ttfamily,keywordstyle=\slshape,language=[LaTeX]{TeX}}
%\begin{lstlisting}[style=tt]
%	\documentclass{foo}
%\end{lstlisting}

%% URL
%      
\usepackage{url}
\usepackage{breakurl}
\usepackage{color}
\usepackage[dvipsnames]{xcolor}

\usepackage{tikz}


% Farben
%
\usepackage{xcolor}

%% PDF
%
\usepackage{pdfpages}

%% Autonum Funktion
%
\usepackage{ifthen}       


%% Querverweise
\usepackage{appendix}
\usepackage{varioref}
\usepackage[
colorlinks,        % Links ohne Umrandungen in zu wählender Farbe
linkcolor=black,   % Farbe interner Verweise
filecolor=black,   % Farbe externer Verweise
citecolor=black,    % Farbe von Zitaten
urlcolor=black
]{hyperref}
\usepackage{cleveref}


%% Code